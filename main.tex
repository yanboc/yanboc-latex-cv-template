% LaTeX 中文简历模板(多字体&主题色/本硕博通用)
% Github: https://github.com/yanboch/LaTeX-Chinese-CV-Template
% 
% 使用方法(具体参见 README.md):
% 1. 使用 XeLaTeX 编译
% 2. 请在 config.tex 中配置个性化内容
% 3. 模块信息请在 includefiles 文件夹中编辑
% 
% 致谢: 
% - 本模板的大框架基于
%   1. [北师大中文CV模板](https://github.com/LeyuDame/BNUCV)
%   2. [西北工业大学中文CV模板](https://www.overleaf.com/latex/templates/npu-cv/mncqzxhvfzrx)

\documentclass[11pt]{article}
\usepackage{xltxtra}
\usepackage{hyperref}
\hypersetup{hidelinks}
\usepackage{bookmark}
\urlstyle{tt}
\usepackage{multicol}
\usepackage{xcolor}
\usepackage{calc}
\usepackage{graphicx}
\usepackage{tikz}
\usetikzlibrary{calc, backgrounds}
\usepackage{xeCJK}
\usepackage{relsize}
\usepackage{xspace}
\usepackage{fontawesome5}   
\usepackage{titlesec}
\usepackage{enumitem}
\usepackage{siunitx}
\usepackage{amssymb}
\usepackage{tabularx}
\usepackage{fancybox}
\usepackage{float}
\usepackage{xifthen}

% 注入配置文件
%%%%%%%%%%%%%%%%%%%%%%%%%%%%%%%%%%%%%%%%%%%%%%%%%%%%%%%%%%%%%%%%%%%%%%%%%%%%%%%%%%%%%%%%%%%%%%%%%%%%
%%%%%%%%%%%%%%%%%%%%%%%%%%%%%%%%%%%%%%%%%%%%%%%%%%%%%%%%%%%%%%%%%%%%%%%%%%%%%%%%%%%%%%%%%%%%%%%%%%%%
% 1. 个人信息

% 学校、院系、专业等
\newcommand{\schoolNameCH}{武汉大学} % 学校中文名
\newcommand{\schoolNameEN}{Wuhan University} % 学校英文名
\newcommand{\departmentNameCH}{计算机学院} % 院系中文名
\newcommand{\departmentNameEN}{School of Computer Science} % 院系英文名

% 联系方式
\def\needEmail{true} % 设置为 false 时将不显示邮箱
\newcommand{\yourEmail}{yourEmail@example.com}
\def\needPhone{true} % 设置为 false 时将不显示手机号
\newcommand{\yourPhone}{130-6666-0000}
\def\needGithub{true} % 设置为 false 时将不显示GitHub
\newcommand{\yourGithub}{yanboc}
\def\needWechat{true} % 设置为 false 时将不显示微信
\newcommand{\yourWechat}{yourWechatID}

% 基本信息
% 更多个人信息的添加/修改请在 includefiles/personal_information.tex 中进行
\newcommand{\yourName}{你的名字}
\newcommand{\yourCity}{家乡/学校}
\newcommand{\yourBirthdate}{1900年1月}
\newcommand{\yourContact}{\yourPhone} % 默认联系方式与手机号相同,可以自行修改
\newcommand{\yourExpectedGraduationTime}{2077年7月}
\newcommand{\yourResearchInterest}{你的研究兴趣}

%%%%%%%%%%%%%%%%%%%%%%%%%%%%%%%%%%%%%%%%%%%%%%%%%%%%%%%%%%%%%%%%%%%%%%%%%%%%%%%%%%%%%%%%%%%%%%%%%%%%
%%%%%%%%%%%%%%%%%%%%%%%%%%%%%%%%%%%%%%%%%%%%%%%%%%%%%%%%%%%%%%%%%%%%%%%%%%%%%%%%%%%%%%%%%%%%%%%%%%%%
% 2. 模块设置(标题、内容、图标等)
% 
% 接下来的部分不一定每个都需要;请根据需要将一些true改为false
% 比方说如果你不需要“技能特长”这一栏,那就将
% \def\needSkills{true} -> \def\needSkills{false}
% 如需更多模块请自行添加或在ISSUE留言

% 个人信息模块
\def\needPersonalInformation{true} % 是否需要个人信息模块,true/false
\def\PersonalInformationTitle{个人信息}
\def\PersonalInformationIcon{\faAddressCard}
\def\needAvatar{true} % 是否需要头像,true/false
\def\AvatarWidth{0.15} % 头像宽度,推荐0.12~0.15,与个人信息模块宽度之和小于0.95
\def\PersonalInformationWidth{0.78} % 个人信息模块宽度
\def\AvatarImage{images/avatar.png} % 头像图片路径

% 教育背景
\def\needEducation{true} % 是否需要教育背景模块,true/false
\def\EducationTitle{教育背景}
\def\EducationIcon{\faGraduationCap}

% 科研成果
\def\needPublication{true} % 是否需要科研成果模块,true/false
\def\PublicationTitle{科研成果}
\def\PublicationIcon{\faBook}

% 项目/实习
\def\needProjects{true} % 是否需要项目/实习模块,true/false
\def\ProjectsTitle{项目与实习}
\def\ProjectsIcon{\faTools}

% 技能特长
\def\needSkills{true} % 是否需要技能特长模块,true/false
\def\SkillsTitle{技能特长}
\def\SkillsIcon{\faWrench}

% 竞赛经历
\def\needCompetitions{false} % 是否需要竞赛经历模块,true/false
\def\CompetitionsTitle{竞赛经历}
\def\CompetitionsIcon{\faTrophy}

% 所获荣誉
\def\needHonors{true} % 是否需要所获荣誉模块,true/false
\def\HonorsTitle{所获荣誉}
\def\HonorsIcon{\faCertificate}

% 其他
\def\needOthers{true} % 是否需要其他模块,true/false
\def\OthersTitle{其他}
\def\OthersIcon{\faInfo}

%%%%%%%%%%%%%%%%%%%%%%%%%%%%%%%%%%%%%%%%%%%%%%%%%%%%%%%%%%%%%%%%%%%%%%%%%%%%%%%%%%%%%%%%%%%%%%%%%%%%
%%%%%%%%%%%%%%%%%%%%%%%%%%%%%%%%%%%%%%%%%%%%%%%%%%%%%%%%%%%%%%%%%%%%%%%%%%%%%%%%%%%%%%%%%%%%%%%%%%%%
% 3. 个性化配置(简易版,更多细节的修改可在 main.tex 中进行)

% 页眉与校徽与主题色
\def\useDefaultTheme{true} % 是否使用默认主题,详见 README.md
\def\useSchoolLogo{true} % 是否使用学校logo,true/false
\def\useSchoolName{false} % 是否使用学校名称,true/false
\def\schoolLogo{images/school_logo.png} % 校徽图片路径
\def\schoolLogoWidth{0.15} % 校徽宽度,占行宽的比例(根据logo宽度调整)

% 参考主题色(这里给出红绿蓝三种,其他颜色可自定)
% 武大配色方案参考:https://www.whu.edu.cn/xxgk/wdbs.htm
% 隔壁配色方案参考:https://vi.hust.edu.cn/jcbf/scgf/scxl.htm
\definecolor{WHU_Green}{HTML}{115740}
\definecolor{WHU_Blue}{HTML}{002554}
\definecolor{PKU_Red}{HTML}{94070A}
\def\themeColor{WHU_Blue} % 如果使用默认主题,请勿更改

% 水印设置
\def\needWatermark{true} % 是否需要水印,true/false
\def\watermarkOpacity{0.03} % 水印透明度,0-1
\def\watermarkImage{images/school_watermark.png} % 水印图片路径
\def\watermarkSize{0.9} % 水印大小,0-1

% 字体设置
% 参见 README.md -> 自定义字体
\def\useDefaultFont{false} % 是否使用默认字体,详见 README.md
\def\fontPath{fonts/} % 字体文件路径
\def\customFontFamilyCH{SweiSpring} % 如果使用默认字体,请勿更改
\def\fontFileTypeCH{.ttf} % 字体文件类型,.ttf/.otf
\def\fontBoldFontCH{* Bold} % 粗体字体文件名
\def\customFontFamilyEN{\customFontFamilyCH} % 如果使用默认字体,请勿更改
\def\fontFileTypeEN{\fontFileTypeCH} % 字体文件类型,.ttf/.otf
\def\fontBoldFontEN{\fontBoldFontCH} % 粗体字体文件名

% 模块标题格式
\newcommand{\moduleTitle}[2]{
    \vspace{1em}
    {\LARGE\bfseries
    {\color{\themeColor}{#1}} \hspace{0.5em} {#2}} \vspace{-0.6em}
    \par
    \vspace{0.5em}
    {\color{\themeColor}\rule{\textwidth}{0.8pt}}
    \par
}
\newcommand{\nextPage}{
    \newpage
    \ifthenelse{\boolean{\needWatermark}}{\watermark}{}
    }

%%%%%%%%%%%%%%%%%%%%%%%%%%%%%%%%%%%%%%%%%%%%%%%%%%%%%%%%%%%%%%%%%%%%%%%%%%%%%%%%%%%%%%%%%%%%%%%%%%%%
%%%%%%%%%%%%%%%%%%%%%%%%%%%%%%%%%%%%%%%%%%%%%%%%%%%%%%%%%%%%%%%%%%%%%%%%%%%%%%%%%%%%%%%%%%%%%%%%%%%%
% 模板设置
% 参考 https://github.com/LeyuDame/BNUCV/tree/main

% 取消中文字符与数字之间的间隔
\setlength{\parindent}{0pt} % 取消全局段落缩进
\pagenumbering{gobble} % 取消页码显示
\setlist[itemize]{topsep=0em, leftmargin=*} % 增加了itemize顶部间距
\setlist[enumerate]{topsep=0em, leftmargin=*} % 增加了enumerate顶部间距
\setlist[description]{topsep=0em, leftmargin=*} % 增加了description顶部间距

% 行间距与字间距
\renewcommand{\arraystretch}{1.2} % 1.2倍表格行间距
\linespread{1.2} % 1.2倍正文行间距
\renewcommand{\CJKglue}{\hskip 0.05em} % 中文字符间距

% 页面设置
\usepackage[a4paper, % 纸张大小为A4
    left=1.2cm, right=1.2cm, top=1.8cm, bottom=1.2cm, % 页边距
    headsep=14pt, footskip=14pt % 页眉页脚间距
]{geometry}

% 页眉页脚设置
\usepackage{fancyhdr}
\pagestyle{fancy}
\fancyhf{} % 清空默认页眉页脚
\renewcommand{\headrulewidth}{0pt}
\renewcommand{\footrulewidth}{0pt}

\newcommand{\pureColorHeader}{
    \node[anchor=north, inner sep=0pt](header) at (current page.north){
        \tikz[inner sep=0pt]{\fill[\themeColor] (0,0) rectangle (\paperwidth, 0.072\paperwidth);}
        };
}
\newcommand{\pureColorFooter}{
    \node[anchor=south, inner sep=0pt](footer) at (current page.south){
        \tikz[inner sep=0pt]{\fill[\themeColor] (0,0) rectangle (\paperwidth, 0.042\paperwidth);}
    };
}

% 页眉(校徽(左)+院系名(右))
\fancyhead[C]{
    \begin{tikzpicture}[remember picture, overlay]
        \ifthenelse{\boolean{\useDefaultTheme}}{
        \node[anchor=north, inner sep=0pt](header) at (current page.north){
            \includegraphics[width=\paperwidth]{images/header.png}
        };
        }{\pureColorHeader}
        % 插入学校logo(默认插入,需要添加学校logo的图片并在config.tex中配置\schoolLogo路径)
        \ifthenelse{\boolean{\useSchoolLogo}}{
            \node[anchor=west](school_logo) at (header.west){
                    \hspace{0.5cm}
                    \includegraphics[width=\schoolLogoWidth\textwidth]{\schoolLogo}
                };

        }{}
        % 插入学校名称(默认不插入,使用学校logo中的校名)
        \ifthenelse{\boolean{\useSchoolName}}{
            \node[anchor=west](school_name_text) at (header.west){
                \hspace{0.5cm}
                \textcolor{white}{\textbf{\schoolNameCH}}
            };}{}
        \node[anchor=east](department_name_text) at(header.east){
            \textcolor{white}{\textbf{\departmentNameCH | \departmentNameEN}}
            \hspace{0.5cm}
        };
    \end{tikzpicture}
    \vspace{-3.5em}
}

% % 页脚,联系方式
\fancyfoot[C]{
    \begin{tikzpicture}[remember picture, overlay]
        \ifthenelse{\boolean{\useDefaultTheme}}{
        \node[anchor=south, inner sep=0pt](footer) at (current page.south){
            \includegraphics[width=\paperwidth]{images/footer.png}
        };
        }{\pureColorFooter}
        \node[anchor=center] at(footer.center){\contact};
    \end{tikzpicture}
}

\thispagestyle{fancy}

% 自定义字体,详见 README.md -> 自定义字体
\ifthenelse{\boolean{\useDefaultFont}}{}
{
    % 英文字体
    \setmainfont[
        Path={\fontPath},
        Extension={\fontFileTypeEN},
        BoldFont={\fontBoldFontEN},
    ]{\customFontFamilyEN}
    % 中文字体
    \setCJKmainfont[
        Path={\fontPath},
        Extension={\fontFileTypeCH},
        BoldFont={\fontBoldFontCH},
    ]{\customFontFamilyCH}
}
%%%%%%%%%%%%%%%%%%%%
% 个人信息
%%%%%%%%%%%%%%%%%%%%
% 联系方式
\newcommand{\contact}{
    % 根据个人喜好选择页脚的字号
    % \small % 小
    % \footnotesize % 更小
    \scriptsize % 再小一号

    \textcolor{white}{
        % 邮箱
        \ifthenelse{\boolean{\needEmail}}{
            \faEnvelope \quad \href{mailto:\yourEmail}{\yourEmail}
        }{} 
        \hspace{4em}
        % 手机号
        \ifthenelse{ \boolean{\needPhone}}{
            \faPhone \quad \yourPhone
        }{}
        \hspace{4em}
        % GitHub
        \ifthenelse{ \boolean{\needGithub}}{
            \faGithub \quad \href{https://github.com/\yourGithub}{\yourGithub}
        }{}
        \hspace{4em}
        % 微信
        \ifthenelse{ \boolean{\needWechat}}{
            \faWeixin \quad \yourWechat
        }{}
    }
}

% 水印
\newcommand{\watermark}{
    \begin{tikzpicture}[remember picture, overlay]
        \node[opacity=\watermarkOpacity] at(current page.center){
            \includegraphics[width=\watermarkSize\paperwidth, keepaspectratio]{\watermarkImage}
        };
    \end{tikzpicture}
}

\begin{document}
\ifthenelse{\boolean{\needWatermark}}{\watermark}{}

% 个人信息
\vspace{-1.2em}
\newcommand{\infoGap}{\ \ \ \ \ \ \ \ }
\ifthenelse{\boolean{\needPersonalInformation}}{
    \ifthenelse{\boolean{\needAvatar}}{
        \begin{figure}[h]
            \begin{minipage}{\PersonalInformationWidth\textwidth}
                \moduleTitle{\PersonalInformationIcon}{\PersonalInformationTitle}
                \begin{tabularx}{\linewidth}{p{\widthof{四个汉字:}}Xp{\widthof{四个汉字:}}X}
                    姓 \infoGap 名: & \yourName & 所在城市: & \yourCity  \\
出生年月: & \yourBirthdate & 联系方式: & \yourContact \\
毕业时间: & \yourExpectedGraduationTime & 研究兴趣: & \yourResearchInterest 

% 如需修改请仿照上述格式添加,例如添加政治面貌与性别:
% 政治面貌: & \yourPoliticalStatus & 性 \infoGap 别: & \yourGender \\
% 其中,\infoGap 为两个空格,对齐用。\\为换行符,最后一行不用加。
                \end{tabularx}
            \end{minipage}
            \hspace{2em}
            \begin{minipage}{\AvatarWidth\textwidth}
                \setlength{\fboxsep}{0pt}
                \doublebox{\includegraphics[width=0.99\textwidth]{\AvatarImage}}
            \end{minipage}
        \end{figure}
    }{
        \begin{figure}[h]
                \moduleTitle{\PersonalInformationIcon}{\PersonalInformationTitle}
                \begin{tabularx}{\linewidth}{p{\widthof{四个汉字:}}Xp{\widthof{四个汉字:}}X}
                    姓 \infoGap 名: & \yourName & 所在城市: & \yourCity  \\
出生年月: & \yourBirthdate & 联系方式: & \yourContact \\
毕业时间: & \yourExpectedGraduationTime & 研究兴趣: & \yourResearchInterest 

% 如需修改请仿照上述格式添加,例如添加政治面貌与性别:
% 政治面貌: & \yourPoliticalStatus & 性 \infoGap 别: & \yourGender \\
% 其中,\infoGap 为两个空格,对齐用。\\为换行符,最后一行不用加。
                \end{tabularx}
            \hspace{2em}
        \end{figure}
    }
}

\vspace{-1em}

% 教育背景
\ifthenelse{\boolean{\needEducation}}{
    \newcommand{\addEducation}[6]{
        {\large \textbf{#1}},{#2} \hfill {#3} \\
        \underline{#4},专业:{#5} \hfill {#6}
        \par \vspace{0.8em}
    }
    \newcommand{\addCourse}[1]{\vspace{-0.8em} #1 \par \vspace{0.8em}}
    \newcommand{\addEducationDescription}[1]{\vspace{-0.5em} #1 \par \vspace{0.8em}}
    \moduleTitle{\EducationIcon}{\EducationTitle}
    % 请按照:
% \addEducation{学校名称}{学位/学历}{学校位置}{学院}{专业}{起止时间}
% \addCourse{主修课程:课程1、课程2、课程3、课程4等。}
% \addEducationDescription{主修课程/GPA/综测(本科)/研究方向/导师(研究生)}
% 的格式添加教育背景。

\addEducation{武汉大学}{本科}{湖北,武汉}{弘毅学堂}{数学与应用数学}{2016年9月--2020年6月}
\addCourse{\textbf{主修课程}:高等概率论、泛函分析、拓扑学。注:可以用\underline{下划线}或\textbf{加粗}来突出重点。暂时不支持斜体。}
\addEducationDescription{\textbf{综合评价}:1. \textbf{GPA}: 4.00/4.00,2. \textbf{综测排名}: 1/120,在校期间曾参加$\cdots$竞赛,获评$\cdots$称号。(这里可以选择性地加一行对本科阶段的简评,突出一下成绩排名、主要获奖等。)}

\addEducation{武汉大学}{硕士(如不需要可以整条删掉,下同)}{湖北,武汉}{数学与统计学院}{基础数学}{2020年9月--2022年6月}
\addCourse{\textbf{主修课程}:非参统计、高维统计、机器学习等。}
\addEducationDescription{\textbf{综合评价}:1. \textbf{GPA}: 4.00/4.00,2. \textbf{综测排名}: 1/120(如果不需要这一行可以直接删掉。)}

\addEducation{武汉大学}{博士}{湖北,武汉}{计算机学院}{计算机科学与技术}{2022年9月--至今}
\addEducationDescription{\textbf{研究兴趣}:1. \textbf{AI安全}:研究LLM对齐的相关理论与应用,在$\cdots$出版物上发表论文$\cdots$篇。2. \textbf{推理优化}:研究LLM优化原理,包括$\cdots$等。这里主要写个大概,具体内容可以在“科研成果”部分展开。如果在硕士或本科阶段有科研经历也可以加上。}
\addEducationDescription{\faLightbulb \quad 使用说明:总之,这个模块的模板就是一个行间距较小的“加入课程”($\backslash \rm{addCourse}\{\}$)与一个行间距较大的“加入描述”($\backslash \rm{addEducationDescription}\{\}$),你也可以用来写任何你希望出现在这里的东西)。}
    \par
}{}

% 科研成果
\ifthenelse{\boolean{\needPublication}}{
    \newcommand{\addPublication}[5]{
        #1\\
        #2 \hfill 
        \textbf{#3}(#4)\underline{#5} 
        \par \vspace{0.8em}
    }
    \newcommand{\addPublicationDescription}[1]{\par\vspace{-0.5em}#1\vspace{0.8em}}
    \moduleTitle{\PublicationIcon}{\PublicationTitle}
    % 请按照:
% \addPublication{论文标题}{发表时间}{作者}{会议/期刊}{状态}{备注(如发表物等级、奖项等)}
% \addPublicationDescription{论文简介} (可选)
% 的格式添加科研成果。

\addPublication{An Example of Your Publication that Have Been Published.}{\textbf{Your Name}, Author 1, Author 2, Your Supervisor.}{Publication 1}{已发表}{}

\addPublication{Another Example of Your Accepted Paper Not Published Yet.}{Author 1*, \textbf{Your Name*}, Your Supervisor.}{Publication 2}{已接收}{备注,如等级、贡献等}

\addPublication{You Can Add Description Below: Anything You Want is Okay.}{\textbf{Your Name}, Author 1, Your Supervisor.}{Publication 3}{在投}{}
\addPublicationDescription{\faLightbulb \quad  使用说明:这里可以加入对文章的描述,如主要结论、文章获奖等。如果文章信息过长,也可以在这里手动添加信息,注意排版一致性即可。}
    \par
}{}

% 项目经历\科研经历\项目与教学(标题请根据需要修改)
\ifthenelse{\boolean{\needProjects}}{
    \newcommand{\addProject}[4]{
        {\large{\textbf{#1}}} \hfill {#2}\\
        \textbf{#3} \hfill #4
        \par \vspace{0.8em}
    }
    \newcommand{\addProjectDescription}[1]{\par\vspace{-0.5em}#1\par\vspace{0.8em}}
    \moduleTitle{\ProjectsIcon}{\ProjectsTitle} 
    % 请按照:
% \addProject{项目名称}{项目状态、类型}{你在项目中扮演的角色}{项目时间} 的格式添加项目基本信息。
% \addProjectDescription{项目简介。} 的格式添加项目简介。
% 例如:
% \addProject{华为-武大xxx数据库开发}{横向/纵向-已完结/进行中}{开发者}{2020年8月--2021年8月}
% \addProjectDescription{项目简介。}

\addProject{华为 -- AI工程师}{实习}{端边协同推理开发}{2025年5月至8月}
\addProjectDescription{该项目拟在2B/2H场景下实现端边协同推理,即在昇腾910B+310P/B服务器上基于MiniCPM实现VQA、长视频理解、RAG等功能。本人负责维护MiniCPM-o的LLM backbone并在端、边侧设备搭建PP框架,实现了chunked prefill、prefix cache等功能并进行图编译/算子级优化。}
\addProjectDescription{\textbf{相关技能}:熟悉docker、linux系统运维、熟悉昇腾加速卡特性(下至算子级)与调试调优手段(使用profiler分析性能并优化算子、类torch.compile编译加速方法)、有从0到1构建项目(装配物理机、搭建环境、开发代码与测试、调优与维护)的经验。}

% 手动分页(不建议使用 \newpage ,会有对齐问题)
% 加在任何你觉得应该分页的地方
\nextPage

\addProject{华为 -- 昇思 MindSpore 框架开发(线上)}{实习}{MindNLP 开发}{2025年1月至6月}
\addProjectDescription{该实习为华为昇思 MindSpore 深度学习框架开发 NLP 套件(对标 HuggingFace 的 Transformers 库),本人在其中参与了两个模型迁移的任务。}
\addProjectDescription{\textbf{相关技能}:熟悉 Python 模组(类 transformers)的开发流程(编码规范、单元测试、提交PR等全流程)。}

\addProject{\faGithub\ AI Infra 开源仓库贡献}{开源项目}{贡献者}{2025年9月至今}
\addProjectDescription{该项目由 B 站 UP 主 ZOMI 酱牵头,拟以在线合作的形式撰写关于 AI Infra 的教材。本人在其中负责撰写有关集合通信原理、PyTorch 实现、集群互联等部分。}

\addProject{\faGithub\ Qwen Code CLI UI 开源项目(10+ stars)}{开源项目}{主导者}{2025年9月至今}
\addProjectDescription{该项目旨在为接入 Qwen Code CLI 并为之提供图形界面。后期开发计划包括连接微信,方便用户用手机指挥 AI 在服务器上帮自己跑代码。}

    \par
}{}

\ifthenelse{\boolean{\needSkills}}{
    \newcommand{\addSkill}[1]{\item #1}
    \moduleTitle{\SkillsIcon}{\SkillsTitle}
    \begin{itemize}
        % 请按照:
% \addSkill{技能名称} 的格式添加技能。
% 例如:
% \addSkill{英语:六级800分、托福200分;}

\addSkill{\textbf{英语}:无障碍阅读、写作、日常交流(六级578分、托福104分)}
\addSkill{\textbf{编程}:熟练使用
\begin{itemize}
    \item \textbf{深度学习框架}:熟悉 Python + Torch \& MindSpore 开发、调试全流程,熟悉常见大模型内部结构,可基于 Torch 手搓 3D 并行。
    \item \textbf{推理引擎}:会使用 vLLM、MindIE 服务化部署 LLM,了解主流调度系统的实现。
    \item \textbf{推理优化}:熟悉 Torch \& MindSporse 图模式的原理与实践,有算子开发经验。熟悉常见推理阶段量化、caching、调度、PTDES 并行等技术。
\end{itemize}
}
    \end{itemize}
    \par
}{}

% 竞赛经历
\ifthenelse{\boolean{\needCompetitions}}{
    \moduleTitle{\CompetitionsIcon}{\CompetitionsTitle}
    \newcommand{\addCompetition}[4]{
        \textbf{#1} & #2 & #3 & #4 \\
    }
    \begin{table}[h!]
        \begin{tabularx}{\textwidth}{Xp{\widthof{xxxxxxxxxx}}p{\widthof{xxxxxxxxxxxxxxx}}p{\widthof{xxxxxxxxxxx}}}
            % 请按照:
% \addCompetition{比赛名称}{比赛角色}{比赛成绩}{比赛时间/地点} 的格式添加竞赛经历。

\addCompetition{全国大学生数学建模竞赛}{队长}{湖北省一等奖}{2023年9月}
\addCompetition{ACM区域赛}{队员}{铜奖}{2024年9月}
        \end{tabularx}
    \end{table}
    \par \vspace{-2em}
}{}

% 所获荣誉
\ifthenelse{\boolean{\needHonors}}{
    \moduleTitle{\HonorsIcon}{\HonorsTitle}
    \newcommand{\addHonor}[1]{\item #1}
    \begin{multicols}{2}
        \begin{itemize}
            % 请按照:
% \addHonor{荣誉名称} 的格式添加荣誉。
% 例如:

\addHonor{\textbf{乙等优秀学生奖学金}(¥3000, 2023年)}
\addHonor{\textbf{甲等优秀学生奖学金}(¥6000, 2024年)}
\addHonor{\textbf{滴滴“e鸣博士生论坛”}(二等奖,¥5000,2024年)}
\addHonor{\textbf{腾讯奖学金}(¥12000,2024年)}
        \end{itemize}
    \end{multicols}
}{}

% 其他
\ifthenelse{\boolean{\needOthers}}{
    \moduleTitle{\OthersIcon}{\OthersTitle}
    \newcommand{\addOther}[1]{\item #1}
        \begin{itemize}
            % 请按照:
% \addOther{其他名称} 的格式添加其他。
% 例如:

\addOther{
    \textbf{使用说明}: 参见项目文件夹中的 README.md 文件。
}
\addOther{
    \textbf{致谢}: 本模板主要参考下面的两个项目:
    \begin{itemize}
        \item \href{https://github.com/LeyuDame/BNUCV}{北师大中文CV模板(https://github.com/LeyuDame/BNUCV)}
        \item \href{https://github.com/npu-cv/npu-cv}{西北工业大学中文CV模板(https://github.com/npu-cv/npu-cv)}
    \end{itemize}
    我加入了一些低代码化的适配,并加入了更多自定义选项。
}
\addOther{
    \textbf{联系作者}:如果你需要加入其他模块,或者在使用中遇到问题,请通过以下方式联系我:
    \begin{enumerate}
        \item 在本仓库提 Issues 或 PR(推荐)
        \item 直接在小红书或使用邮箱联系我
    \end{enumerate}

    \vspace{0.3em}
    {\Large\textbf{感谢你对本模板的支持!祝你秋招/暑研/面试/样样都顺利!}}
}
        \end{itemize}
        \par
}{}

\end{document}
