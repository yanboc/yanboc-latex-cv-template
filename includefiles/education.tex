% 请按照:
% \addEducation{学校名称}{学位/学历}{学校位置}{学院}{专业}{起止时间}
% \addCourse{主修课程:课程1、课程2、课程3、课程4等。}
% \addEducationDescription{主修课程/GPA/综测(本科)/研究方向/导师(研究生)}
% 的格式添加教育背景。

\addEducation{武汉大学}{本科}{湖北,武汉}{弘毅学堂}{数学与应用数学}{2016年9月--2020年6月}
\addCourse{\textbf{主修课程}:高等概率论、泛函分析、拓扑学。注:可以用\underline{下划线}或\textbf{加粗}来突出重点。暂时不支持斜体。}
\addEducationDescription{\textbf{综合评价}:1. \textbf{GPA}: 4.00/4.00,2. \textbf{综测排名}: 1/120,在校期间曾参加$\cdots$竞赛,获评$\cdots$称号。(这里可以选择性地加一行对本科阶段的简评,突出一下成绩排名、主要获奖等。)}

\addEducation{武汉大学}{硕士(如不需要可以整条删掉,下同)}{湖北,武汉}{数学与统计学院}{基础数学}{2020年9月--2022年6月}
\addCourse{\textbf{主修课程}:非参统计、高维统计、机器学习等。}
\addEducationDescription{\textbf{综合评价}:1. \textbf{GPA}: 4.00/4.00,2. \textbf{综测排名}: 1/120(如果不需要这一行可以直接删掉。)}

\addEducation{武汉大学}{博士}{湖北,武汉}{计算机学院}{计算机科学与技术}{2022年9月--至今}
\addEducationDescription{\textbf{研究兴趣}:1. \textbf{AI安全}:研究LLM对齐的相关理论与应用,在$\cdots$出版物上发表论文$\cdots$篇。2. \textbf{推理优化}:研究LLM优化原理,包括$\cdots$等。这里主要写个大概,具体内容可以在“科研成果”部分展开。如果在硕士或本科阶段有科研经历也可以加上。}
\addEducationDescription{\faLightbulb \quad 使用说明:总之,这个模块的模板就是一个行间距较小的“加入课程”($\backslash \rm{addCourse}\{\}$)与一个行间距较大的“加入描述”($\backslash \rm{addEducationDescription}\{\}$),你也可以用来写任何你希望出现在这里的东西)。}