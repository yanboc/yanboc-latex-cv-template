% 请按照:
% \addProject{项目名称}{项目状态、类型}{你在项目中扮演的角色}{项目时间} 的格式添加项目基本信息。
% \addProjectDescription{项目简介。} 的格式添加项目简介。
% 例如:
% \addProject{华为-武大xxx数据库开发}{横向/纵向-已完结/进行中}{开发者}{2020年8月--2021年8月}
% \addProjectDescription{项目简介。}

\addProject{华为 -- AI工程师}{实习}{端边协同推理开发}{2025年5月至8月}
\addProjectDescription{该项目拟在2B/2H场景下实现端边协同推理,即在昇腾910B+310P/B服务器上基于MiniCPM实现VQA、长视频理解、RAG等功能。本人负责维护MiniCPM-o的LLM backbone并在端、边侧设备搭建PP框架,实现了chunked prefill、prefix cache等功能并进行图编译/算子级优化。}
\addProjectDescription{\textbf{相关技能}:熟悉docker、linux系统运维、熟悉昇腾加速卡特性(下至算子级)与调试调优手段(使用profiler分析性能并优化算子、类torch.compile编译加速方法)、有从0到1构建项目(装配物理机、搭建环境、开发代码与测试、调优与维护)的经验。}

% 手动分页(不建议使用 \newpage ,会有对齐问题)
% 加在任何你觉得应该分页的地方
\nextPage

\addProject{华为 -- 昇思 MindSpore 框架开发(线上)}{实习}{MindNLP 开发}{2025年1月至6月}
\addProjectDescription{该实习为华为昇思 MindSpore 深度学习框架开发 NLP 套件(对标 HuggingFace 的 Transformers 库),本人在其中参与了两个模型迁移的任务。}
\addProjectDescription{\textbf{相关技能}:熟悉 Python 模组(类 transformers)的开发流程(编码规范、单元测试、提交PR等全流程)。}

\addProject{\faGithub\ AI Infra 开源仓库贡献}{开源项目}{贡献者}{2025年9月至今}
\addProjectDescription{该项目由 B 站 UP 主 ZOMI 酱牵头,拟以在线合作的形式撰写关于 AI Infra 的教材。本人在其中负责撰写有关集合通信原理、PyTorch 实现、集群互联等部分。}

\addProject{\faGithub\ Qwen Code CLI UI 开源项目(10+ stars)}{开源项目}{主导者}{2025年9月至今}
\addProjectDescription{该项目旨在为接入 Qwen Code CLI 并为之提供图形界面。后期开发计划包括连接微信,方便用户用手机指挥 AI 在服务器上帮自己跑代码。}
